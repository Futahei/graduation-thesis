% ■ アブストラクトの出力 ■
%	◆書式:
%		begin{jabstract}〜end{jabstract}	:日本語のアブストラクト
%		begin{eabstract}〜end{eabstract}	:英語のアブストラクト
%		※ 不要ならばコマンドごと消せば出力されない.


% 日本語のアブストラクト
\begin{jabstract}

本論文では,Web 検索における検索結果一覧画面でそれぞれの情報の鮮度が一目で分かるシステムを提案する.

一般的な Web ブラウザでは,検索した情報が公開されてからの経過時間に応じて表示に差を設けてはいない.しかし,情報の取捨選択において情報の鮮度は重要な要素の一つである.

そこで,各情報の表示に,実世界における情報媒体が劣化していくメタファを適用する.これによりユーザは,検索した情報の鮮度をより直感的に認識できる.

\end{jabstract}
