\chapter{序論}
\label{chap:introduction}

本章では、本研究の背景と本論文の構成について述べる。

\section{背景}

知りたい情報を探す方法として、現在最も利用されているのはインターネットだろう。

特に Web ブラウザでの情報検索は社会生活を送る中で必要不可欠と言ってもいいかもしれない。かくいう私も Web 検索は常日頃から行っている人間の一人である。

一般的に情報の質を考えるとき、古い情報よりも新しい情報の方が価値が高いことが多い。しかしながらインターネット上に存在する情報は、実世界に存在する紙やインクのような記録媒体に比べて時間経過による外見的な劣化がないため、一目で情報載鮮度が分からない。

そこで、本来外見的な劣化のないデジタルデータに時間経過による表示の変化を与え、すぐに情報の鮮度を認識できるようにすることでより効率的に Web 検索による情報収集を行いことができると考えた。

\section{本研究の目的}

ブラウザにおける検索結果一覧画面を拡張することで、ユーザがより直感的に情報の鮮度を認識できるようにすることが本研究の目的である。

\section{本文書の構成}

第\ref{chap:introduction}章では本研究における背景と目的について書いた。

第\ref{chap:verification}章では、実際に視覚化システムを開発する前に様々な視覚化の方法を試し考察した。

第\ref{chap:latex}章で図表や数式の挿入など代表的な\LaTeX コマンドを解説する。

第\ref{chap:conclusion}章では、『序論』で始めたら『結論』で終われと書いた手前書かざるを得ないので、なにか結論らしいことを書く。付録として、テンプレートのサンプルになるように無理矢理ゴミを添付する。
