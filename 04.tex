\chapter{議論}
\label{chap:discussion}

本章では,第\ref{chap:implementation}章で開発した拡張機能を実際に運用した結果と評価について述べ,それを踏まえて議論を行う.

\newpage

\section{筆者の運用}

筆者は本拡張機能を開発段階から3か月程度利用した.利用している中で,Web 上には古い情報が大量に存在していると確認できた.これは,新しい情報と古い情報がすべて均一に並べられていたために気づかなかったことである.

また,全体を一覧した際にどの情報の鮮度がいいのか直感的に識別可能で,新しい情報が欲しい際にすぐに見つけられた.

今回の拡張機能では,\ref{subsec:ver-tex-sheet}で示した紙の劣化の表現方法を採用したが,他の表現方法にも可能性を感じている.

例えば,\ref{sec:ver-font}で示したフォントを変更する表現方法は,良いフォントを選定できれば実用性がある.他にも,\ref{sec:ver-character}で示した文章が消失する方法は,「消える=古い」という結びつけが出来ればよい.

ユーザごとに相性の良い表現方法を選択可能にするため,本拡張機能に鮮度の表現方法を選択する機能を追加することも視野に入れている.

\section{第三者の意見}
\label{sec:dis_third}

研究会のメンバーの一人に本システムを実際に利用してもらい,フィードバックをもらった.

古い情報が見えにくくなる点については,検索する分野によってはその分野の情報全体が古い場合があり,外見的な劣化が全体におよんでしまうという意見があった.

しかし同時に,情報の鮮度が視覚化されることによって通常の Web 検索に比べて,見ている情報の古さを意識できるようになったと評価された.

\section{今後の展望}

検証,実装および運用を経て,二つの問題点が浮かび上がった.

\begin{itemize}
  \item 情報の分野ごとに全体の鮮度の平均が違う
  \item 正確な情報の鮮度を取得できないことがある
\end{itemize}

それぞれに関しての考察と展望を以下に述べる.

\subsubsection{分野ごとの鮮度}

情報の分野によっては全体の鮮度が古い場合が存在する.例えば,既に更新が止まっている昔の技術を利用しようと検索をした場合,出てくる情報はいずれも古くなるのが必然である.

あるいは情報がすくない場合,一覧されるの情報の鮮度もバラツキが生まれ,外見的な劣化を加えることで本来よりも知りたい情報にたどり着きにくくなってしまう恐れがある.

そこでユーザが自由に,鮮度の算出を行う際の閾値を設定できるようにし,かつ,検索ワードごとの閾値のおススメを表示できるようにすれば,本システムの利点を残したまま,検索精度をあげることができる可能性がある.

\subsubsection{正確な鮮度の取得}

本システムでは検索結果一覧に並んでいる Google Chrome が表示している各項目の更新日時を参考にして鮮度を算出している.

しかし,Chrome の表示する更新日時は正確ではないこともあり,場合によっては表示されていないこともある.本システムでは更新日時が表示されていない情報に関しては,最も古い情報として扱っている.

他に更新日時を知る方法として該当リンク先のページで document.lastModified を実行して,時刻データを入手する方法がある.しかしこの方法はページの仕様によっては,コードを実行した時の時刻が返ってきてしまう.

筆者が調べた限りでは,現状全てのサイトごとの正確な鮮度を知る方法はなく,各サイトごとの更新者が申請する以外にない.それらを集めたデータベースなどを作る方法が考えられるが現実的とは言い難い.

この問題は今後の課題としたい.
