\chapter{議論}
\label{chap:discussion}

本章では、第\ref{chap:implementation}章で開発した拡張機能を実際に運用した結果と評価について述べ、それを踏まえて議論を行う。

\section{筆者の運用}

筆者は本拡張機能を開発段階から3か月程度利用した。利用している中で感じたのは、Web 上には古い情報が大量に存在しているということである。

新しい情報と古い情報がすべて均一に並べられていたという事実を今まで意識してこなかったことが痛感できた。また、全体を一覧した際にどの情報の鮮度がいいのかをすぐに識別可能で、新しい情報が欲しい際などにすぐに見つけることができた。

\section{第三者の意見}

研究会のメンバーの一人に本システムを実際に利用してもらい、フィードバックをもらった。

古いものが見えにくくなる点については、検索する分野によってはその分野の情報全体が古い場合があり、外見的な劣化が全体におよんでしまうという意見があった。

しかし同時に、情報の鮮度が視覚化されることによって通常の Web 検索に比べて、見ている情報の古さを意識できるようになったと評価された。

\section{関連研究}

廃れるリンク\cite{dyinglink}は Web 上のリンクに対して「モノが廃れる」メタファを適用することで、リンクの鮮度を一目で判断させること目的としたシステムである。

本研究との違いは、適用するメタファの選定や鮮度の算出部分の他に、こちらはブラウザにおける検索結果一覧に絞ったものという点である。

特に鮮度の算出に関して、廃れるリンクではプロキシサーバーを利用した大規模のものになったいるが、本研究ではローカルの Chrome 拡張のみで動作することができるため、導入も容易である。

\section{今後の展望}

検証、実装および運用を経て、二つの問題点が浮かび上がった。

\begin{itemize}
  \item 情報の分野ごとに全体の鮮度の平均が違う
  \item 正確な情報の鮮度を取得できないことがある
\end{itemize}

それぞれに関しての考察と展望を以下に述べる。

\subsubsection{分野ごとの鮮度}

情報の分野によっては全体の鮮度が古い場合が存在する。例えば、既に更新が止まっている昔の技術を利用しようと検索をした場合、出てくる情報はいずれも古くなるのが必然である。

あるいは情報がすくない場合、それぞれの情報の鮮度もバラツキが生まれたどり着きたい情報にもたどり着きにくくなってしまう恐れがある。

そこでユーザが自由に鮮度の算出を行う際の閾値を設定できるようにし、かつ、検索ワードごとの閾値のおススメを表示できるようにすれば、本システムの利点を残したまま、検索精度をあげることができる可能性がある。

\subsubsection{正確な鮮度の取得}

本システムでは検索結果一覧に並んでいる Google Chrome が表示している各項目の更新日時を参考にして鮮度を算出している。

しかし、Chrome の表示する更新日時は正確ではないこともあり、場合によっては表示されていないこともある。本システムでは更新日時が表示されていない情報に関しては、最も古いものとして扱っている。

他に更新日時を知る方法として該当リンク先のページで document.lastModified を実行して、時刻データを入手する方法である。しかしこの方法はページの仕様によっては、コードを実行した時の時刻が返ってしまう。

筆者が調べた限りでは、現状全てのサイトごとの正確な鮮度を知る方法はなく、各サイトごとの更新者が申請する以外にない。それらを集めたデータベースなどを作る方法が考えられるが現実的とは言い難い。

この問題は今後の課題としたい。
